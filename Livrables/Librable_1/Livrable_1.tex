\documentclass{article}
\usepackage[utf8]{inputenc}
\usepackage[T1]{fontenc}
\usepackage[french]{babel}
\usepackage{amsmath}
\usepackage{amssymb}
\usepackage{array}
\usepackage{booktabs}
\usepackage{geometry}
\geometry{a4paper, margin=1in}

\title{Livrable 1 : Organisation de l'Équipe et Plan de Travail}
\author{ENSAM RABAT}
\date{\today}


\begin{document}

\maketitle

\section*{Membres de l'Équipe (Trinôme)}

Afin de structurer le travail et d'assurer une couverture complète des modules complexes de ce projet, nous proposons la répartition des rôles suivante :



\maketitle

\section*{Composition de l'Équipe (Trinôme)}

Le projet sera mené par l'équipe suivante, avec une répartition claire des responsabilités pour les différents modules :

\begin{table}[h]
    \centering
    \caption{Répartition des rôles et des responsabilités}
    \begin{tabular}{>{\raggedright\arraybackslash}m{3cm} >{\raggedright\arraybackslash}m{3cm} >{\raggedright\arraybackslash}m{7cm}}
        \toprule
        \textbf{Rôle} & \textbf{Membre} & \textbf{Missions Principales et Modules Cibles} \\
        \midrule
        \textbf{Chef d'Équipe / NLP Lead} & \textbf{Zaynab ER-REGHAY} & Coordination générale, gestion du Git, intégration des modules NLP et fusion (Modules 1, 4, 5). \\
        \midrule
        \textbf{CV Lead / Gabarits Expert} & \textbf{Lahfari BILAL} & Prétraitement des images/PDF, développement du Module Gabarits, implémentation et entraînement du modèle CV hybride (Modules 1, 2, 3). \\
        \midrule
        \textbf{Architecte Système / MLOps} & \textbf{Malek SAMI} & Architecture offline, pipeline principal (\texttt{main.py}), interface utilisateur, gestion des dépendances et optimisation des performances (Modules 1, 5, 6). \\
        \bottomrule
    \end{tabular}
\end{table}

\hrule
\vspace{0.5cm}

\section*{ Plan de Travail Détaillé (Phase 1 \& 2)}

Le projet sera découpé en deux phases principales pour assurer une livraison progressive et des tests d'intégration réguliers, conformément aux recommandations du sujet\textsuperscript{1}.

\subsection*{Phase 1 : Infrastructure \& Prototypes Simples (Jours 1-10)}

L'objectif est de mettre en place l'environnement 100\% offline et de disposer d'un prototype fonctionnel pour chaque tâche de classification.

\begin{table}[h]
    \centering
    \caption{Détail de la Phase 1}
    \label{tab:phase1_detail}
    \begin{tabular}{m{0.5cm} m{6cm} m{2cm} m{4cm}}
        \toprule
        \textbf{Étape} & \textbf{Tâches Clés} & \textbf{Responsable} & \textbf{Livrables de la Phase 1} \\
        \midrule
        1. & Config Offline : Création de \texttt{setup\_offline.py} (téléchargement ResNet50, CamemBERT, Tesseract) & Architecte / CV Lead & \multirow{2}{4cm}{Script \texttt{setup\_offline.py}} \\
        2. & Définition de la structure de dossiers \texttt{models/} & & \\
        \midrule
        3. & Pipeline Basique : Conversion PDF vers images (Module 6, Étape 1) & \multirow{2}{2cm}{Architecte} & \multirow{2}{4cm}{Classe \texttt{OfflineModelManager}} \\
        4. & Implémentation de la classe \texttt{OfflineModelManager} (chargement simple des modèles)\textsuperscript{6} & & \\
        \midrule
        5. & Prototype NLP (Motifs) : Définition des dictionnaires de motifs sémantiques pour les 5 classes\textsuperscript{7} & \multirow{2}{2cm}{NLP Lead} & \multirow{2}{4cm}{Dictionnaires de motifs (JSON/Python)} \\
        6. & Implémentation du score de classification basé uniquement sur les mots-clés (4.2.1)\textsuperscript{8} & & \\
        \midrule
        7. & Prototype CV (ResNet) : Entraînement initial simple de ResNet50 (sans gabarits) pour établir une baseline\textsuperscript{9} & CV Lead & Baseline ResNet50 simple. \\
        \bottomrule
    \end{tabular}
\end{table}

\hrule
\vspace{0.5cm}

\section*{ Stratégie de Données Synthétiques}

Étant donné l'impossibilité de collecter un jeu de données réelles suffisant et labellisé (CNIE, relevés bancaires, factures, etc.), nous allons adopter une stratégie de \textbf{Génération de Données Synthétiques} combinée à des techniques d'Augmentation de Données pour contourner ce blocage.

\subsection*{1. Génération de Données Synthétiques}

L'objectif est de créer des documents qui imitent la structure visuelle et le contenu textuel des 5 classes, sans utiliser de vraies données sensibles.

\subsubsection*{Pour les Relevés/Factures (Structure Tabulaire) :}
\begin{itemize}
    \item Utiliser des bibliothèques de génération de PDF pour créer des tables avec des champs typiques (Montant, Date, Description).
    \item Remplir les champs avec des données aléatoires mais cohérentes (ex: dates logiques, montants float).
    \item Intégrer les \textbf{Motifs Sémantiques} définis (ex: "solde", "kWh", "m³") dans les textes générés\textsuperscript{10}.
\end{itemize}

\textit{Intérêt} : Fournit un corpus labellisé pour le fine-tuning de CamemBERT et une base d'entraînement pour la détection de structure tabulaire (Transformée de Hough)\textsuperscript{11}.

\subsubsection*{Pour les Pièces d'Identité (Format et Features Gabarits) :}
\begin{itemize}
    \item Générer des images avec le ratio d'aspect correct (format carte)\textsuperscript{12}.
    \item Placer des zones pour la photo, le numéro d'identité, et la date, pour entraîner les détecteurs de zones (ex: Cascade Classifiers pour la photo)\textsuperscript{13}.
    \item Ajouter des images de fond (comme la carte du Maroc) pour simuler les gabarits visuels\textsuperscript{14}.
\end{itemize}

\subsection*{2. Augmentation de Données (Clé de la Robustesse)}

Nous appliquerons des techniques d'augmentation agressives aux données synthétiques pour simuler les conditions réelles des documents administratifs\textsuperscript{15}.

\begin{itemize}
    \item \textbf{Augmentations CV} : Rotations légères, changements de contraste, ajout de bruit, compression JPEG pour simuler une mauvaise qualité de scanner/photo\textsuperscript{16}.
    \item \textbf{Augmentations OCR} : Dégrader artificiellement les images (flou, faible résolution) pour entraîner le pipeline à mieux gérer les erreurs post-OCR\textsuperscript{17}.
\end{itemize}

\subsection*{3. Validation}

Un petit jeu de données réelles (anonymisées et ne nécessitant pas d'autorisation spéciale) sera recherché pour la \textbf{Validation Finale} et l'évaluation de la Robustesse\textsuperscript{18}.

\end{document}